\section*{\centering Abstract}
\addcontentsline{toc}{subsection}{Abstract}
%The impact of health, particularly on human capital development and upholding human right, has been emphasized for several decades. Despite the global commitment to universal health exemplified in the past millenium development goals (MDGs) and the ongoing sustainable development goals (SDG), health problems remains a significant challenge in the developing countries. Some attributed such development deficit to lack of enough resources in the poor nations, making a case for foreign aid (ODA) and the need for its effectiveness. Yet past studies assessing the ODA effectiveness on health have often employed narrow health indicators. Against this backdrop, the thesis assesses the effectiveness of ODA using comprehensive composite health dimensions from SGD 2.4 on malnutrition and SDG 3 on health. The thesis also sought to understand the mediating role of social protection in such impact. The study employs a combination of panel and dynamic econometric approaches, Specifically the novel approach fixed effect cross-lag panel model (FE-CLPM), unit fixed effect and local projection, with 144 countries data between 2000 to 2021, mean aggregated into 5 periods with both total net ODA and social infrastructure ODA. Despite the variations observed in the ODA impact across health dimensions, all models reveals both total and social infrastructure ODA has significant effect on reproductive fatality, infection and diseases as well as environmental death. Moreover, findings reveals no evidence that ODA improves health system capacity nor does the impact of ODA across all health dimensions mediated by social protection or different across regions. Disparities in studies, including this thesis, could be attributed econmetrics approache employed, control variables, specific health indicators and type of ODA employed. Therefore, the thesis concludes that ODA, particularly when targeted to social infrastructure ODA, significantly influences health outcome. Enhancing health system capacity in developing countries should be the paramount focus of global development effort. This could be ensured by strengthening ODA monitoring mechanisms as well as holding recipient government responsible for mismanagement.


%Key words: Official Development Assistance (ODA or Foreign Aid), Social Protection, Health Outcomes, dynamic panel model. 





The imperatives of health in human capital development and the promotion of human rights has been emphasized for decades. Despite global commitments to universal health, as evident in the Millennium Development Goals (MDGs) and the ongoing Sustainable Development Goals (SDGs), health challenges persist in developing countries. This deficit is often attributed to resource limitations in poorer nations, prompting a reliance on Foreign Aid (ODA) to address these issues. However, previous studies assessing ODA's impact on health have predominantly utilized narrow indicators. Against this backdrop, this thesis evaluates ODA effectiveness by employing comprehensive composite health dimensions created from SDG 2 (malnutrition) and SDG 3 (health and well-being). Additionally, it explores the mediating role of social protection in this impact. Utilizing a combination of panel econometric approaches, including the fixed effect cross-lag panel model (FE-CLPM), unit fixed effect, and local projection, the study analyzes data from 144 countries spanning 2000 to 2021, mean aggregated into five periods, considering both total net ODA and social infrastructure ODA. Despite observed variations in ODA impact across health dimensions, all models reveal significant effects of both total and social infrastructure ODA on reproductive fatality, infection and diseases, as well as environmental death. Findings, however, suggest no evidence that ODA improves health system capacity, nor does its impact across health dimensions mediated by social protection or differ significantly between SSA and non-SSA regions. Discrepancies in study outcomes, including this thesis, may be attributed to differences in econometric approaches, control variables, specific health indicators, and types of ODA employed. Consequently, the thesis concludes that ODA, especially when directed towards social infrastructure, significantly influences health outcomes. Emphasizing the enhancement of health system capacity in developing countries should be a primary focus of global development efforts. This necessitates strengthening ODA monitoring mechanisms and holding recipient governments accountable for mismanagement.

\vspace{10pt}
\textbf{Keywords}: Official Development Assistance (ODA or Foreign Aid), Social Protection, Health Outcomes, Composite health dimension, Dynamic Panel Model, Fixed Effect Cross-Lag Panel Model (FE-CLPM), 