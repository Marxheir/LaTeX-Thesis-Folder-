The third classification of variables is covariates, also known as control variables. In this study, covariates are confounding factors affecting both ODA and health outcomes. They mitigate omitted variable bias (OVB) in the model and ensure baseline equivalence among study units. All control variables, except the Climate Risk Index (CRI) are sourced from the World Development Indicators \parencite{wdi_world_2023}. The following are the controlled variables considered in this study:

\begin{enumerate}[i]
    \item \textbf{GDP per Capita:} GDP per capita is fundamental to accounting for the initial level of economic development in a country. An increase in countries' GDP Per capita enhances their health outcome and reduces their likelihood of receiving ODA \parencite{oecd_ODA_Report_2023}. This variable, used in numerous reviewed studies \parencite[e.g.,][]{williamson_foreign_2008, nwude_impact_2023, toseef_how_2019}, is measured in 2017 constant price US dollars \textcite{wdi_world_2023}. The variable is logged in the model to address high skewness.
   % \item \textbf{Government Health Spending:} Domestic government health expenditure signifies the level of a country's commitment to its healthcare system. This spending not only influences the strength of the healthcare system but also impacts health outcomes, as used in previous studies \parencite{yan_mortality_2015, chung_economic_2022, mohamed_foreign_2017, akinola_foreign_2022}. The variable is measured as a percentage of total health spending in 2017 constant price \textcite{wdi_world_2023} and logged in the model.  
    \item \textbf{Domestic government health finance capacity:} The thesis employs both domestic government health spending per capita and stock of external debt, as a proxy for countries' public finance. While government health spending per capita signifies the level of a country's commitment to its healthcare system, \parencite[see][]{yan_mortality_2015, mohamed_foreign_2017, akinola_foreign_2022}, the external debt stock exerts downward pressure government's capacity to invest in the health system \parencite{kumar2010public, ogunjimi2019impact}. Surprisingly, prior studies did not control for government debt, despite its significant influence on public health finance in developing countries \parencite{kumar2010public}. Government spending per capita is measured in 2017 constant price, external debt is measured as per GNI \parencite{wdi_world_2023}. Both variables are logged to mitigate skewness.   
    
   % \item \textbf{Log(External Debt Stock):} A crucial factor influencing the public finance and investment capacity of many developing countries is public debt. In an IMF-published study, \textcite{kumar2010public} revealed a negative relationship trend between the external public debt-to-GDP ratio and economic growth. Specifically, a 10 percentage point rise in the initial debt-to-GDP ratio led to a 0.2 percentage point annual reduction in real per capita GDP growth \parencite{kumar2010public}. A more detailed negative impact of external debt is uncovered in a study by \textcite{ogunjimi2019impact} in Nigeria. Surprisingly, prior studies did not control for government debt, despite its significant influence on public health finance in developing countries. The variable is measured as a percentage of the country's GNI and sourced from the World Development Indicators \parencite{wdi_world_2023}. The thesis applies a logarithmic transformation to the variable to mitigate the impact of skewness caused by extreme debt-ridden countries.
    
    \item \textbf{Demographic factors:} Country size, proxy by population, and population density are crucial to countries' health outcomes. Countries with higher populations often strain healthcare systems. Population density may support healthcare infrastructure development, enhancing accessibility to medical facilities, but may also contribute to the spread of infectious diseases and inadequate sanitation, negatively affecting health. While  \textcite[e.g.,][]{staicu2017study, chung_economic_2022, yogo_health_2015, doucouliagos_health_2021} controlled for population as a proxy of country size, \textcite{doucouliagos_health_2021} employed both population and population density. Both variables are logged in the models.
 %   
  %  Higher populations often strain healthcare systems and correlate with increased health problems. This variable is measured in absolute values and sourced from the World Development Indicators \parencite{wdi_world_2023}. Despite its high correlation with other demographic variables, the population is retained due to its significance. The logarithmic transformation is applied to address skewness. Noteworthy, this variable exhibits a positive effect on all health dimensions, with a specific positive impact on HSCR.
   % \item \textbf{Log(Population Density):} Another vital social demographic factor influencing health is population density, in accordance with previous studies \parencite{toseef_how_2019, doucouliagos_health_2021}. The thesis controls for population density, measured as people per square kilometer of land area, with data sourced from the World Development Indicators \parencite{wdi_world_2023}. To address skewness, a log transformation is applied.
    %Higher population density can have a dual impact: it may support healthcare infrastructure development, enhancing accessibility to medical facilities, but it can also contribute to the spread of infectious diseases and inadequate sanitation, negatively affecting health. Consequently, the thesis anticipates a negative association of population density with all health dimensions, while expecting a positive effect on health system capacity and responsiveness.
    
    \item \textbf{Infrastructure development:} The level of infrastructure development plays a crucial role in the overall health and well-being of citizens. While \textcite{doucouliagos_health_2021, yogo_health_2015} controlled for sanitation and water access, \textcite{odokonyero_impact_2018, yan_mortality_2015, williamson_foreign_2008} employed urbanization to measure infrastructure development. As utilized by \textcite{toseef_how_2019} and due to the high correlation of electricity access with various development indicators, including education, sanitation and water access, HDI etc, the thesis employs electricity access as a proxy of the infrastructure development level of countries.  Electricity access is measured as the percentage of the population with access to electricity \textcite{wdi_world_2023}. 
    
    \item \textbf{Alternative foreign inflow:} Foreign Direct Investment (FDI) and remittance are crucial alternative foreign inflows in developing countries \parencite{scott_lessons_2020}, capable of bolstering investment capital and strengthening health systems and outcomes. The nature of impact of all foreign inflows depends on the recipients of those flows: while remittance flows to households \parencite[see][]{obi2020does, gamlen2014new, de2009remittances}, FDI encourages flow technology \parencite{chung_economic_2022} and ODA flow more to the government. Regardless, all foreign inflows, regardless of their nature, have an enormous impact on health outcomes and the well-being of developing countries, thus making FDI and remittances crucial in the thesis models.   
    
    %While \textcite{yogo_health_2015} controlled for remittance per capita, remittances typically don't directly contribute to government funds, limiting their impact on public health investment and overall national development \parencite[see][]{obi2020does, gamlen2014new, de2009remittances}. In alignment with \textcite{chung_economic_2022}, this thesis opts for FDI as an alternative foreign financial flow. The variable is measured as net inflows as a percentage of GDP, with data sourced from the World Development Indicators \parencite{wdi_world_2023}.
    
    \item \textbf{Climate Risk Index Score:} The Climate Risk Index (CRI) Score ranks countries and regions based on their vulnerability to the impacts of climate change, particularly extreme weather events like storms, heatwaves, and floods \parencite{eckstein2021global}. The escalating impact of climate change on global health cannot be overstated. While \textcite{nwude_official_2020} controlled for CO$_2$ emissions, the thesis contends that the impact of climate change on the health of developing countries' populations is not necessarily proportional to their emissions. Hence, the thesis opts for the CRI Score, which reflects the vulnerability and exposure of developing countries to various climate impacts, likely affecting population health. The CRI Score, compiled by German Watch, ranges from 2 to 166 and is sourced from the most recent version in 2021 \parencite{eckstein2021global}. 
    
    %\item \textbf{Log(Trade per GNI):} Trade is a fundamental factor influencing the economic development, health, and overall well-being of a country. Nations with higher trade tend to exhibit positive economic and health outcomes \parencite[see][]{makki2004impact, pavcnik2017impact}. Previous studies incorporated trade in various forms, such as export value in \textcite{chung_economic_2022} and net trade values in \textcite{forte_impact_2023, shafiullah_foreign_2011}, as a proxy for countries' trade openness. In a similar vein, this thesis controls for countries' level of trade, measured as a percentage of GDP. The variable undergoes a log transformation, and raw data is sourced from the World Development Indicators \parencite{wdi_world_2023}. Trade is anticipated to have a negative effect on all health dimensions but a positive impact on health system capacity and responsiveness.
    
    \item \textbf{Quality of governance:} The impact of institutional/governance quality continues to be debated \parencite[e.g.][]{doucouliagos_health_2021, williamson_foreign_2008}. Previous studies have employed various governance proxies, such as control of corruption, Fraser freedom index, civil conflict, political regimes, and democracy index sourced from the World Development Indicators \parencite{shafiullah_foreign_2011, muhammad_health_2021, toseef_how_2019, williamson_foreign_2008, kavakli_us_2022, yogo_health_2015}. In this thesis, governance quality is proxied by the general Governance Index, which allows inter-country comparisons across six crucial governance dimensions: voice and accountability, political stability and absence of violence, government effectiveness, regulatory quality, rule of law, and control of corruption \parencite[see][for detailed description of the index]{handoyo2023worldwide, wdi_world_2023}.
\end{enumerate}









     
       






 
 

