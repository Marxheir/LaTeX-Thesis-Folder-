\section*{\centering Chapter Six}
\section*{\centering Discussion of Findings, Conclusions and Recommendations}
%\addcontentsline{toc}{section}{Chapter Six}
\addcontentsline{toc}{section}{Chapter Six: Discussion of Findings, Conclusions and Recommendations}
The chapter presents the discussion of key findings from the analyses and further contextualizes them within the existing studies. Furthermore, it draws major conclusions and relevant policy implications. The chapter ends with study limitations and suggestions for future studies. 
\subsection*{6.1 Discussion of Findings}
\addcontentsline{toc}{subsection}{6.1 Disscusion of Findings}
\subsubsection*{\quad 6.1.1 Impact of ODA on Health outcomes}
\addcontentsline{toc}{subsubsection}{6.1.1 Impact of ODA on Health outcomes}
The overarching objective of this thesis is to evaluate the impact of Official Development Assistance (ODA) on health outcomes in developing countries. According to the findings, the impact of ODA vary based on ODA type, specific health dimensions, and the chosen econometric approach. Despite this variability, the analysis reveals that ODA has notable impact on specific health dimensions. Both the unit fixed effect and Fixed Effect Cross-lag Panel  Models (FE-CLPM), for instance, demonstrate that ODA significantly reduces reproductive fatality (RFTP), burden of Infection and Diseases (BID), and environmental deaths. These findings are further strengthened in the local projection model, where ODA demonstrates both intermediate and long-term significant effects on these three health dimensions. 
While these composite health dimensions are unique to this study, prior research has consistently shown similar findings on specific indicators aggregated within them. Using adult death—embedded in reproductive fatality and environmental, for instance, death—\textcite{yan_mortality_2015} reveals that the Global Health Fund had a significant effect on mortality rates in developing countries. Comparable findings are reported by \textcite{doucouliagos_health_2021} and \textcite{kavanagh_governance_2019} on the significant impact of ODA on child mortality, a component embedded in the RFTP health dimension.

Moreover, a study on the impact of the Mexico City Policy (MCP), aimed at reducing USA ODA on reproductive health, by \textcite{kavakli_us_2022} reported a significant rise in maternal and child mortality, as well as HIV incidence—crucial indicators in the burden of infection and diseases (BID) and RFTP health dimensions. These findings are supported by a range of other studies \parencite{akinola_foreign_2022, muhammad_health_2021, yogo_health_2015}, all of which reveal the positive impact of ODA on respective health indicators aggregated in the thesis's health dimensions. Contrary findings, however, have been reported in numerous other studies \parencite{williamson_foreign_2008, nwude_official_2020, bavinger_relationship_2017, toseef_how_2019}. These findings bear relevance to this study because the specific indicators employed are embodied in the thesis's health dimensions, particularly BID, environmental death, as well as RFTP.

Notably, one reason for disparities among studies could be attributed to the lag period of ODA. Since ODA does not have an immediate effect on health, studies with unlagged ODA are less likely to reveal the desired result due to model misspecification. As shown in the local projection Figure \ref{fig:Local_projection}, the causal response of ODA on health dimensions manifests significantly after four years following allocation. Previous studies, such as those by \textcite{kavanagh_governance_2019}, \textcite{kavakli_us_2022}, \textcite{chung_economic_2022}, and \textcite{akinola_foreign_2022}, employed lagging ODA with a first-order lag, while \textcite{doucouliagos_health_2021} used a five-year lag, consistent with the approach adopted in this thesis and revealed similar findings.

In addition, the FE-CLPM model shows both total and social infrastructure ODA have significant positive impact on malnutrition. It's noteworthy that this effect is distinctive to the FE-CLPM model and is not observed in other models. This findings is novel, given the limited existing research exploring this specific area.
Moreover, the analysis indicates that none of the types of Official Development Assistance (ODA) has the desired impact on Health System Capacity and Responsiveness (HSCR). Surprisingly, all coefficients of ODA on this health dimension were consistently negative across models. While studies examining the influence of ODA on health system capacity in developing countries are limited, a qualitative study by \textcite{cassola_evaluating_2022} suggests that ODA has little or no effect on the health research landscape in developing countries, aligning with the findings of this thesis. This alignment may contribute to the perspectives of aid pessimists, including \textcite{easterly_aid_2004}, who posit that the inefficacy of foreign aid stems from its emphasis on treatment rather than prevention.  Furthermore, none of the models yield noteworthy findings on the impact of ODA on the burden of mental problems (BMP). Comparison with other studies on this health dimension is challenging due to the elusive nature of mental health. Moreover, this pattern of findings and the lack of study could also be a consequence of the lower incidence of mental problems in underdeveloped countries, which are the major recipients of ODA.




\subsubsection*{\quad 6.1.2 Regional Heterogeneity of ODA Impact on Health}
\addcontentsline{toc}{subsubsection}{6.1.2 Regional Heterogeneity of ODA Impact on Health}

The study reveals inconsistent evidence regarding regional variations in the impact of ODA on health outcomes, as both types of total ODA only vary across region for certain health dimensions. Specifically, in the main model, the impact of total ODA on burden of infection and diseases and reproductive fatality are significant in MENA while reproductive fatality and environmental death is significant in Europe. However, none of the coefficient of ODA is significant across all health dimensions for SSA region. Surprisingly, in the robustness FE-CLPM, social infrastructure ODA is found to be weakly significant for Reproductive Fatality and Teen Pregnancy (RFTP), Burden of Infection and Diseases (BID), and malnutrition for East Asia and Pacific (EAP) and Middle East and North Africa (MENA) regions. However, it is mildly significant only on RFTP and malnutrition for the SSA region. 


Additionally, among all the regions, social infrastructure ODA is only positive and mildly significant on health system capacity (HSCR) for EAP region. Due to inconsistency in the result, the study finds no evidence that the impact of ODA significantly vary among the regions. While these findings may be controversial, similar results have been reported. Utilizing only health ODA, \textcite{nwude_official_2020} found no regional variation in the impact of ODA on life expectancy between SSA and non-SSA regions. Although life expectancy is not part of the thesis health dimensions, health ODA is a subsector of social infrastructure ODA used in this thesis. Moreover, across other studies \parencite{gama_health_2015, staicu2017study, shafiullah_foreign_2011} with regional comparing found no significant difference in the impact of ODA on education, income, and inequality—all of which are essential predictors of health outcomes, between SSA and non-SSA regions.


\subsubsection*{\quad 6.1.3  Mediating Role of Social Protection in the Impact of ODA on Health Outcome}
\addcontentsline{toc}{subsubsection}{6.1.3 Mediating Role of Social Protection}

In exploring the impact of ODA on various health dimensions, this thesis assesses the possible mediating role of social protection. Surprisingly, the indirect effect of social protection insignificant, with a confidence interval encompassing zero for all health dimensions. Thus, social protection does not mediate any effect of ODA across all health dimensions. While this form of mediation analysis is novel in the assessment of ODA effectiveness on health, making comparisons with existing studies challenging.

It is crucial to note, however, that these findings do not necessarily imply that ODA does not influence social protection, as indicated by \textcite{nino-zarazua_aids_2023}, or that social protection does not affect health outcomes, as evidenced by studies such as \textcite{sepehri_how_2014, sarkodie_effect_2021, heggebo_disentangling_2020, adato_social_2009}. The limitation encountered in social protection data and the assumed causal order may underestimate the social protection indirect effect, thus influencing the thesis findings. Therefore, the specific finding is intended solely as a preliminary policy guide for the optimal allocation of ODA, rather than a definitive conclusion. Nevertheless, \textcite{yogo_health_2015} identified female education and government spending as mediators of ODA impact on health in their study, adding complexity to the understanding of mediating factors in the relationship between ODA and health.


\subsection*{6.2 Conclusions}
\addcontentsline{toc}{subsection}{6.2 Conclusion}

In light of the findings, the thesis concludes that ODA, particularly in the form of social infrastructure and total net ODA, have positive and significant impact on health outcomes. However, this impact is health dimension-specific, with both total and social infrastructure ODA demonstrating a more pronounced effect on infections and diseases, reproductive fatality, and environmental death. There is no also an evidence ODA may have a positive impact on malnutrition, albeit weak. Notably, there is no evidence that ODA enhances the health system capacity and responsiveness, which is crucial to ensuring sustainable health security in developing countries. Furthermore, the significance of ODA is contingent on the health priorities of the region, as revealed in the thesis analysis. Lastly, the thesis does not have any conclusive evidence regarding social protection as a potential mediator for improving health outcomes. 


While this research contributes valuable insights to the body of evidence on foreign aid effectiveness, the inconclusiveness observed in previous studies on ODA and health outcomes may be attributed to variations in specific health indicators, econometric approaches, and the specific types of ODA employed. It is crucial to recognize that there is no universally superior model, as threats to validity can only be mitigated, not eliminated. Thus, understanding the true causal impact of ODA on health necessitates a comprehensive examination of a multitude of evidence, including insights from previous studies in developing countries.

% it is important to acknowledge certain limitations, including the causal assumption, data quality, and model specifications, that may have influenced the results.
  
 
\subsection*{6.3 Policy Recommendations}
\addcontentsline{toc}{subsection}{6.3 Policy Recommendations}
In light of the study's findings, the thesis proposes the following policy recommendations to optimize the impact of global development efforts:
\begin{enumerate}[i]
    \item \textbf{Prioritizing Critical health infrastructure in global development effort:} Shifting the focus of global development efforts towards investing in critical health infrastructure, including healthcare facilities, medical personnel, and preventive healthcare measures is crucial for sustainable global health security. This emphasizes the need for long-term investments in critical health infrastructures to enhance a developing nation's self-reliance in managing health challenges. This approach aligns with the principles of sustainable development by promoting autonomy and resilience in healthcare systems. 
   
    \item \textbf{ODA Allocation Mechanisms Strengthening :} Preliminary findings indicate that ODA allocation pattern does not depend on countries' health situations or needs. Strengthening the capacity of the Development Assistance Committee (DAC) and adopting a result-based financing approach can enhance the effectiveness of ODA by directing funds where they are most needed and can have the greatest impact. Targeted ODA allocation based on countries' health situations and needs will improve the effectiveness of aid in addressing specific health challenges. This recommendation aims to ensure that ODA is allocated efficiently, addressing specific health challenges in recipient countries.
    
    \item \textbf{Optimal  Utilization of ODA by Recipient Governments:} A holistic and strategic approach by recipient governments is essential to realizing the full potential of foreign aid. Thus, recipient governments should adopt a visionary approach, focusing on policies that enhance healthcare systems, poverty alleviation, education, social protection, and health research. Strategic utilization of ODA by recipient governments will enhance absorptive capacity, ensuring that aid is optimally used for sustainable development.  
    
    \item \textbf{Governance and Accountability:} While the impact of governance on ODA effectiveness is beyond the scope of this study, governance emerges as a crucial factor in all models. Governance deficiencies, such as corruption and ineffectiveness, impede ODA effectiveness. Therefore, there is need to address governance deficiencies by implementing international sanctions for governments mismanaging ODA, including prevention from future access or requests for ODA return. This recommendation aims to curb corruption and ineffectiveness, enhancing accountability and ensuring that aid funds are used transparently for their intended purposes. Good governance practices are crucial for the successful implementation and impact of foreign aid
    
    \item \textbf{Focus on Economic Development, Debt Forgiveness, and Conflict Prevention:} Recognizing the interconnected nature of development challenges, the thesis advocates for a comprehensive approach that prioritizes economic development, debt forgiveness, and conflict prevention alongside ODA allocation. Excessive debt and economic instability in developing countries contribute significantly to poverty and poor health outcomes. Therefore, the international community, alongside specific developing countries, should prioritize initiatives that create an environment conducive to sustainable development. This holistic strategy addresses underlying factors and emphasizes the interconnected nature of development issues, ultimately contributing to the improvement of overall well-being.
\end{enumerate}

%Based on the findings, the thesis concludes that ODA, particularly social infrastructure ODA has a significant effect on health outcomes. However, this impact is health dimension-specific. For instance, both total and social infrastructure ODA has more impact on infections and diseases, reproductive fatality, and as well as environmental death. Also, there is no evidence that ODA enhances health system capacity and responsiveness in developing countries.  It is essential to note, that much of the inconclusiveness of previous studies on ODA and health outcomes could be attributed to variation in specific indicators employed, econometric approach, and specific type of ODA employed. Moreover, while the total ODA tends to be more significant for infection and diseases in the SSA region, social infrastructure ODA is more significant on health in non-SSA regions. Based on this, the thesis concludes that the significance of ODA depends on the health priority of the region, as shown across the thesis analysis. Finally, the thesis does not have any conclusive evidence of social protection as a potential mediator for improving health outcomes. 

%Based on the findings, the thesis offers the following policy recommendations: 
%\begin{itemize}
 %   \item As revealed in the preliminary findings, ODA is barely allocated to where they are needed the most. To ensure adequate ODA effectiveness, more mechanisms are needed on the supply side of ODA to ensure ODA is allocated to areas where they are most needed. Such mechanism may be through to enhance the capacity of the Development Assistance Committee (DAC) of the OECD as well as the result-based financing approach.  
  %  \item Moreover, recipient governments and countries need to be more visionary on how they utilize the received ODA. Since deficit in health investment and entrenched poverty are the major causes of poor health, ODA recipient governments should focus their attention on policies that enhance their healthcare system and also reduce poverty. Such a policy approach is an investment in education, social protection, and health research to build essential absorptive power for ODA effectiveness.  
   % \item Furthermore, even though the impact of governance on ODA effectiveness is beyond the scope of this study, it is evident from all the models that governance is essential to any kind of development, particularly health. Governance deficiency like corruption and ineffectiveness hampers ODA effectiveness. Therefore, there is need for international sanctions for governments that mismanage ODA allocated to their respective countries. Such sanction may include prevention from future access to ODA or request for returning the ODA even if disbursed on grant. This will enhance the sense of accountability among recipient governments. 
   % \item No matter the amount of ODA allocated, lack of economic development and excessive debt in developing countries will continue to enhance poverty and poor health outcomes. Therefore, the international community should focus development efforts on investing in critical infrastructure, debt forgiveness, and preventing war and violence in developing countries. 
%\end{itemize}

\subsection*{6.4 Study Limitations and Suggestions for Future Research}
\addcontentsline{toc}{subsection}{6.4 Study Limitations and Suggestions for Future Research}
While this study investigates not only the effectiveness of ODA but also which health dimensions are most influenced by ODA, there are certain limitations. Future studies may delve into each of the thesis health dimensions to gain a deeper understanding of their respective intricacies. An especially noteworthy area for future exploration is discerning why ODA exhibits a negative impact on health system capacity. If sustainable development is a critical factor in ODA allocation, investigating its effect on health system capacity should be prioritized in subsequent studies.

Despite the absence of evidence on ODA effectiveness on mental problems in this thesis, future research endeavors may further investigate this aspect. Moreover, while the study offers preliminary findings on the ineffectiveness of social protection as a potential channel for ODA's impact on health, the lack of adequate panel data for social protection performance indicators and the potential inconsistency in causal order assumptions might have led to an underestimation of the mediating impact of social protection. Hence, future studies might focus on specific country-level micro-analysis, employing household surveys to comprehend how emergency social protection from received ODA has improved health situations.

Furthermore, although the thesis findings were consistent to some extent, they are sensitive to the econometric approach employed. Therefore, there is a need for further research grounded in the dynamic panel method. One such approach is the Fixed Effect Cross-Lag Panel Model based on Maximum Likelihood Estimation (ML-SEM) by \textcite{allison2017maximum}. While the thesis employed this method, future researchers are encouraged to explore more on this new approach. ML-SEM has been verified in various simulations to yield more consistent and efficient estimates \parencite[see][]{moral2019dynamic, becker2023many, allison2017maximum}.

Finally, the study focused on two forms of ODA—total and social infrastructure ODA. Future studies may consider exploring subsector-specific ODA, such as basic health, to gain more nuanced insights into the specific areas of impact.

%The study strives towards understanding not only ODA effectiveness but also on which health dimension is ODA most effective. Future studies may explore each of thesis health dimensions, for deeper understanding of the respective health dimensions. One notable health dimension for future exploration will be to understand why ODA is negative for health system capacity. If sustainable development is crucial in ODA allocation, scrutinizing its impact on health system capacity should be focused in future studies.  Despite the lack of evidence in the thesis on the impact of ODA on the burden of mental problems, future studies may investigate this further.  Moreover, while the study provides preliminary findings on the ineffectiveness of social protection as a possible channel for ODA impact on health, lack of adequate panel data for social protection performance indicators and the possibility of inconsistent causal order assumption might have underestimated the mediating impact of Social protection. Therefore, future studies may focus on specific country micro-analysis, using household surveys, to understand how the emergency social protection from ODA received has improved health situations. Furthermore, although the thesis findings were consistent to some extent, there sensitivity of findings to econometric approach employed. Therefore, there is a need for further research that is more grounded in the dynamic panel method.
%One of such is the Fixed effect cross lag panel model based on the maximum likelihood estimation of \textcite{allison2017maximum}. Although, the thesis employed this method I encourage the future to try this new approach. ML-SEM has been proven to produce more consistent and efficient estimate \parencite[see][]{leszczensky2022deal, becker2023many, allison2017maximum}. Finally, the study employed two forms of ODA; total and social infrastructure ODA, future studies may focus on subsector-specific ODA such as basic health.  