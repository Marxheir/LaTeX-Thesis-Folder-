\section*{\centering Chapter One}
\section*{\centering Introduction}
%\addcontentsline{toc}{section}{Chapter One}
\addcontentsline{toc}{section}{Chapter One: Introduction}
% Your introduction content goes here

\subsection*{1.1 \quad Study Background} 
\addcontentsline{toc}{subsection}{1.1	Background of the Study}

Health plays a crucial role in human capital development, upholding fundamental rights, and fostering economic growth \parencite{yogo_health_2015}. This perspective, especially regarding the centrality of health capital in economic development, has been extensively discussed by endogenous growth theorists \textcite{grossman_concept_1972, ostlin_paying_2004, sen_health_1998} amongst others. They argue that a healthy population is a productive one, emphasizing the reciprocal relationship between health and wealth. Despite the significance of this, developing nations continue to face persistent health challenges \parencite{gama_health_2015}, despite decades of global health initiatives. This has become particularly important in the present era of Sustainable Development Goals (SDGs), with SDG 3 dedicated to advancing Universal Health Coverage (UHC) and the overall well-being of all people.

In recent decades, international organizations such as the International Labour Organization (ILO), World Health Organization (WHO), and World Bank have increasingly highlighted the role of social protection in ensuring health security in developing countries. According to these organizations, social protection acts as a potential tool for poverty eradication and reduction, improving access and affordability, and eliminating financial barriers to health services \parencite{fox_clinical_2015, who_world_2010}. The ILO Social Protection Floor (SPF), for example, includes health as a basic and irreducible right for every human \parencite{ilo_social_2012, ilo_towards_2020}. This perspective aligns with the World Bank's social safety net initiative and WHO's Universal Health Coverage (UHC), highlighting the relevance of social protection to health \parencite{jorgensen_social_2019, lonnroth_beyond_2014, siroka_association_2016, who_world_2010}. In the post-COVID-19 era, the nexus between social protection and health is evident \parencite[see][]{yokobori_roles_2023}. Studies consistently highlight social protection's dual impact on public health, emphasizing a rights-based approach and contributing to health financing through social security and assistance \parencite{hagemejer_role_2013, ilo_towards_2020, macnaughton_decent_2010, yokobori_roles_2023, scheil-adlung_response_2014, scheil-adlung_focusing_2020}.

However, achieving adequate health outcomes demands substantial capital investment, posing challenges for developing nations with limited resources \parencite{world_bank_high-performance_2019}. A World Bank report indicates an annual financing gap of \$176 billion USD for UHC in the world's poorest countries by 2030 \parencite{world_bank_high-performance_2019}. Consequently, foreign aid, or Official Development Assistance (ODA), has become crucial as an alternative investment capital to address these deficits, including health-related challenges \parencite{sachs_case_2014, yogo_health_2015}. 


ODA, together with remittances, and foreign direct investment (FDI), serve as crucial channels for foreign currency flow in developing countries \parencite{scott_lessons_2020}. The OECD defines ODA as concessional loans and grants from affluent nations to foster economic growth and enhance well-being in developing countries \parencite{oecd_ODA_Report_2023} \footnote{Since 2017, the official methodology for ODA classification has changed to only consider grant equivalent flow as ODA \parencite{oecd_ODA_Report_2023}}. The effectiveness of aid in influencing diverse developmental outcomes, including economic growth, health, and social protection, has been a focal point in academic and policy discussions for decades. Debates on aid effectiveness trace back to economists such as \textcite{bauer_dissent_1979, easterly_elusive_2002, milton_foreign_1958, sachs_geography_2001}, with diverse perspectives, who are categorized into aid optimists and pessimists \parencite{kavanagh_governance_2019, nwude_official_2020}.

Aid optimists emphasize capital investment, mobilized through domestic savings and technologies, for government-led developmental purposes. The absence or inadequacy of these resources, encompassing financial, technological, and knowledge aspects, underscores the need for ODA \parencite{sachs_development_2005}. Conversely, pessimists point to persistent poverty in Africa and many parts of South Asia despite decades of ODA \parencite[see][]{yontcheva_macroeconomic_2006}. They argue that foreign aid contributes to uneven income distribution, fosters inequality, endangers governance, creates dependency, and is sometimes exploited as a pretext for foreign invasion \parencite{easterly_can_2003, easterly_are_2007, easterly_can_2009, milton_foreign_1958, easterly_elusive_2002}. Pessimists conclude that foreign aid has no positive effect or may indeed have a negative influence in developing countries \parencite{yontcheva_macroeconomic_2006}.


Subsequent scholars have elaborated on the conditions moderating or mediating aid effectiveness, leading to studies emphasizing non-linearity and context conditionality \parencite[see]{yontcheva_macroeconomic_2006}. Debates on foreign aid (or ODA) persist, especially given the rising awareness of interconnections in development challenges, exemplified by global health issues like COVID-19. This awareness has led to increased bilateral ODA allocation towards global public goods, such as public health \parencite{cepparulo_responses_2016, kenny_official_2020}, and renewed global actions on foreign aid effectiveness \parencite{cassola_evaluating_2022, ogbuoji_aid_2018} \footnote{Global actions on aid effectiveness include the Paris Declaration for Aid Effectiveness 2005, the International Health Partnership Plus 2007, the Accra Agenda for Action 2008, the Busan Partnership for Effective Corporation 2011, the Global Partnership for Effective Development Corporation (GPEDC) 2011, the 2015 Financing for Development conference of Addis Ababa Agenda (AAA), Agenda 2030, Sustainable Development Goals 2015 until 2030, and UHC in 2005 (Ibid).}. In light of these developments, the study assesses the impact of foreign aid on health outcomes and the potential mediating role of social protection in developing countries.

%\vspace{4pt}
\subsection*{1.2 \quad	Problem Statement}
\addcontentsline{toc}{subsection}{1.2 	Problem Statement}
Numerous studies have examined the influence of foreign aid on various development indicators, encompassing economic growth, health outcomes, social protection, and environmental efficiency \parencite{cassola_evaluating_2022, easterly_can_2009, milton_foreign_1958, nino-zarazua_aids_2023, nwude_official_2020, yan_mortality_2015}. However, the existing evidence remains inconclusive, with studies broadly categorized into aid pessimists and optimists \parencite{kavanagh_governance_2019, nwude_official_2020}. Optimists argue for the efficacy of foreign aid, highlighting its potential impact on developing countries' conditions in the absence of such aid \parencite{sachs_case_2014}. Studies supporting this view demonstrate that foreign aid effectively promotes economic growth, either directly or indirectly, by enhancing human development, including health security, education, and poverty reduction \parencite{bavinger_relationship_2017, doucouliagos_health_2021, marty_taking_2017, mohamed_foreign_2017}.

Conversely, aid pessimists assert that aid dependency and undue donor influence negate the anticipated benefits of foreign aid \parencite{easterly_are_2007}. Consequently, some studies reveal the ineffectiveness of foreign aid across various health and social development indicators \parencite{ali_foreign_2020, chung_economic_2022, williamson_foreign_2008}. However, others emphasize nuanced conditions moderating aid effectiveness, such as the quality of governance, economic growth, initial development levels in recipient countries \parencite{kavanagh_governance_2019, ogbuoji_aid_2018, shafa_assessment_2023}, and donor-imposed conditions \parencite{nwude_impact_2023}. Despite ongoing discussions, notable gaps persist in previous studies, raising concerns about the credibility of their findings. 

One significant gap is the reliance on health ODA as the primary predictor of health outcomes, neglecting the broader impact of ODA across various sectors and subsectors. This results in the possible underestimation of aid effectiveness, as health ODA represents only a small fraction of the social infrastructure sector.  Moreover, ODA allocated to other sectors and subsectors like climate change, education, population, and reproductive health may exert both direct and indirect influences on overall population health outcomes (see Appendix A Figure \ref{fig:General ODA classification} for OECD's ODA markers). To address this limitation, the thesis combines both total ODA and the social infrastructure sector, providing a more comprehensive assessment. 

%One significant gap is the predominant reliance on health ODA as the primary predictor of health outcomes in previous studies. This approach overlooks the broader impact of ODA distributed across various sectors and subsectors, resulting in underestimation of aid effectiveness. For example, health ODA is a small fraction of the social infrastructure sector, as shown in Appendix Figure \ref{fig:Sector ODA classification}). Moreover, ODA allocated to other sectors and subsectors like climate change, education, population, and reproductive health may exert both direct and indirect influences on overall population health outcomes (see Appendix A Figure \ref{fig:General ODA classification} for OECD's ODA markers). Since using only health ODA may underestimate the impact of interest, the thesis combines both the total ODA and the social infrastructure sector, rather than exclusively considering the health subsector, for a more comprehensive assessment.

Another gap is the choice of indicators for the endogenous variable (health outcome). Since health situations may vary in nature and intensity across regions and countries, as discussed extensively in Chapter Two, using single health metrics such as maternal mortality, malaria incidence and child mortality, may not align with the public health priorities of ODA recipient countries. Such variation could potentially lead to aid diversion \parencite{ogbuoji_aid_2018}. Aid fungibility and the context-specific nature of health problems might introduce bias when comparing countries solely on narrow health indicators. One country, for instance, may allocate ODA to child immunization, improving child health outcomes, while another may focus on adult health. This diversity may not be adequately captured by using a single indicator. Moreover, some indicators respond to changes in ODA more than others \parencite{doucouliagos_health_2021},  thus, the nature of findings may be constrained by incomprehensive indicators. To address these challenges, the thesis combines various indicators from SDG 3 on health and SDG 2 on malnutrition, to capture variations in health problems and burdens across regions, age groups, and genders. The indicators are used to construct comprehensive composite health dimensions \footnote{Please see Chapter Two for a discussion on health dimensions. The procedure for creating such dimensions is described in Appendix A.2} for a broader perspective on health outcomes across all countries.

Finally, understanding the mechanisms through which ODA impacts health outcomes remains a significant gap in the existing literature. Even among studies revealing the positive effects of ODA on health outcomes, none of these studies have explored how these effects manifest, except \parencite[]{yogo_health_2015}. This becomes particularly crucial in the era of Sustainable Development Goals (SDGs), where comprehending the mechanisms of policy intervention is paramount for optimal resource allocation. Moreover, social protection has become a key policy instrument to advance universal health coverage globally.  Against this backdrop, the thesis aims to address the following research questions:



%\vspace{1pt}

\subsection*{1.3 Research Questions}
\addcontentsline{toc}{subsection}{1.3	Research Questions}

\subsubsection*{1.3.1	Primary Research Questions:}
\addcontentsline{toc}{subsubsection}{1.3.1 	Primary Research Questions}

What is the overall effectiveness and impact of foreign aid, specifically Official Development Assistance (ODA), on health outcomes in developing countries? This overarching research question aims to investigate the comprehensive impact of ODA on diverse composite health dimensions created from relevant SDG 2 and 3 indicators. 
\subsubsection*{1.3.2	Specific Research Questions:}
\addcontentsline{toc}{subsubsection}{1.3.2 	Specific Research Questions:}
\begin{enumerate}
    \item What is the direct impact of ODA on health outcomes? This question seeks to understand the direct influence of ODA on health outcomes without the mediation of any third-party variable.
    \item Are there regional variations in the impact of ODA on health outcomes, particularly in SSA and non-SSA regions? This question aims to understand potential regional-specific effects moderating the effectiveness of ODA on health outcomes. The study seeks to determine whether adjusting for regional characteristics, primarily SSA and non-SSA, affects the pattern of foreign aid impact on various health dimensions. 
    \item Is the impact of ODA on health outcomes mediated by social protection development in developing nations? This question aims to explore whether social protection development plays a mechanistic role in the effect of ODA on health outcomes. The implication is that allocating more ODA to social protection could yield dual benefits in poverty eradication and health security.
\end{enumerate}

%\vspace{1pt} 
\subsection*{1.4	Significance of the study}
\addcontentsline{toc}{subsection}{1.4 	Significance of the study}
This thesis significantly contributes to the fields of health economics, social protection, and development studies, offering fresh insights and advancements in several key areas:
\begin{enumerate}[i]
    \item Knowledge and empirical implications: 
    \begin{itemize}
        \item Conceptual Broadening: While previous research has investigated aid effectiveness on narrow health indicators, this study distinguishes itself by broadening the concept of health outcomes. Unlike previous studies that use narrow indicators as proxies for the complex and multidimensional construct of health outcomes, this research constructs diverse composite health dimensions from SDG 2 on malnutrition and SDG 3 on health and human well-being. By including indicators from SDG 2, such as child stunting, women’s anaemia, and incidence of malnutrition, the study ensures a comprehensive understanding of health outcomes. This approach facilitates inter-regional comparison and enhances the comprehensiveness of health interventions.
    \item Mechanistic Exploration: Furthermore, the study stands out by exploring the mechanistic role of social protection development in the impact of foreign aid on health outcomes. While the interconnections between health and social protection have gained recently gained attention, empirical work in this direction remains limited. This study fills this gap, providing valuable insights into the interplay between ODA, social protection, and health outcomes.
    \item Moreover, the study adopts a novel econometric modeling approach by applying a cross-lag panel dynamic model proposed by \textcite{allison2017maximum, moral2019dynamic}, which is based on a maximum likelihood estimation of a structural equation model as opposed to Arrelano Bond the method of a moment condition approach commonly used in previous studies to unravel the causal relationships of interest, contributing to the methodological advancements in this field.
    \end{itemize}
    
    \item Policy and practical implications: 
    \begin{itemize}
        \item In an era of renewed interest in aid effectiveness, this study’s findings are crucial for providing up-to-date evidence and policy recommendations on the impact of ODA. The thesis's dual focus on health outcomes and the mechanisms of the effects, particularly social protection, provides policymakers with practical guidelines on not only what works but also how this works in developing countries. The study extends its significance to ODA donor countries and multilateral organizations by offering insights into optimal ODA allocation. 
        \item Theoretically, the study’s model, derived from rigorous analysis, has the potential to advance perspectives on how health outcomes are theorized, thereby guiding future studies on the relationships between health outcomes and their predictors.
    \end{itemize}
\end{enumerate}
In summary, this study significantly contributes to academic discourse and offers practical implications for policymakers and organizations involved in health, development, and aid effectiveness.


\subsection*{1.5 Organization of the Thesis}
\addcontentsline{toc}{subsection}{1.5 Organization of the Thesis}

The thesis is structured into six chapters, each serving a distinct purpose.

Chapter One, the Introduction, sets the stage for the study by providing a comprehensive background on health outcomes, social protection, and foreign aid. This section establishes the interconnections between these concepts and articulates the problem statement, highlighting gaps in past studies. The chapter then introduces the overarching research questions and the underlying significance of the thesis. The following chapter, Conceptual Analysis, offers a detailed clarification of three key constructs: ODA, Health Outcome, and Social Protection. Moreover, the chapter presents a preliminary analysis concerning foreign aid allocation, regional characterization of health dimensions, and possible overlaps between the burden of health and aid allocation.

Chapter Three, Literature Review, begins with a conceptual framework, employing a causal chain illustrating the theory of change to explore the channels through which ODA affects health outcomes. This chapter further contextualizes these patterns of relationship within the existing body of knowledge, organized according to the research questions. It ends with a summary and conclusions. The following chapter four, Methodology, details the empirical strategy guiding the research. This includes causal assumptions, identification strategies, and specifications of the dynamic panel models. The chapter then discusses the data and sources.

Chapter Five provides results and interpretation, beginning with a statistical summary of all variables used for the models, followed by model tables of various specifications from Chapter 4. Each model table is accompanied by brief explanations of the results. The chapter also includes robustness tests and summary of the major findings. Finally, Chapter six provides a discussion of results, contextualizing the findings from Chapter 5 within the broader context of past studies. The section is followed by conclusions drawn from the study’s results. The chapter closes with recommendations, classified into two categories: policy recommendations for practical insights for policymakers and suggestions for further studies, proposing avenues for future research.
