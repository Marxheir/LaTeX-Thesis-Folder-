\begin{table}[H]
    \centering
    \caption{Breusch-Pagan Test Results for Heteroskedasticity}
    \begin{tabular}{lccc}
        \toprule
        \textbf{Model} & \textbf{BP Statistic} & \textbf{df} & \textbf{P-value} \\
        \midrule
        Reproductive Health & 169.2 & 13 & $< 2.2 \times 10^{-16}$ \\
        HSCR (Health, Safety, and Child Rights) & 55.348 & 13 & $3.51 \times 10^{-07}$ \\
        Environmental Health & 84.453 & 13 & $1.591 \times 10^{-12}$ \\
        Mental Health & 67.146 & 13 & $2.68 \times 10^{-09}$ \\
        Infectious Disease & 35.373 & 13 & 0.000742 \\
        Nutritional Health & 38.087 & 13 & 0.0002793 \\
        \bottomrule
    \end{tabular}
    \vspace{0.5em} % Add some space before the note
    \caption*{Note: The null hypothesis assumes homoskedasticity, and a low P-value suggests rejection of the null hypothesis, indicating the presence of heteroskedasticity in the model residuals.}
    \label{Tab::Heteroskedict}
\end{table}
